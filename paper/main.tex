\documentclass{llncs}

%%%%%%%%%%%%%%%%%%%
%%% Packages
%%%%%%%%%%%%%%%%%%%
\usepackage{xcolor}
\usepackage[backend=biber, style=lncs]{biblatex}
\addbibresource{ref.bib}
% !TEX TS-program = pdflatexmk

\usepackage{geometry}
\geometry{
  a4paper,         % or letterpaper
  textwidth=15cm,  % llncs has 12.2cm
  textheight=24cm, % llncs has 19.3cm
  hratio=1:1,      % horizontally centered
  vratio=2:3,      % not vertically centered
}
%%%%%%%%%%%%%%%%%%%
%%% General Cmds
%%%%%%%%%%%%%%%%%%%
\newcommand{\todo}[1]{{\color{red}\bfseries{}TODO: #1}}
  
\begin{document}
\title{Quantum Pseudo-Telepathy}
\subtitle{The Magic Square Game}

\author{Luigi Soares \and Roberto Rosmaninho}
\institute{%
  Department of Computer Science, UFMG - Brazil \\
  \email{\{luigi.domenico, robertogomes\}@dcc.ufmg.br}}

\maketitle
% \begin{abstract}
% \end{abstract}

\section{Introduction}
\label{sec:intro}

Telepathy, the ability of transmitting information from one
person's mind to another's, would certainly come in handy in
many situations, right? Unfortunately (or not), (to the best
of our knowledge) telepathy is not a thing. At least, not
according to classical physics. Certain aspects of the
quantum realm, however, provide a way of communication that
for a layman looks as magical as ``true'' telepathy. This
phenomenon is called quantum
\emph{pseudo-telepathy}~\cite{brassard:2005}.


Quantum pseudo-telepathy is observed in many contexts,
usually described in the format of a game: the ``impossible
colouring games''~\cite{brassard:2005, Kochen1975}; the
parity games, in which \(n \geq 3\) players are given
bit-strings and, without communicating to each other, they
output one of their bits, winning if their outputs combined
obey certain parity conditions~\cite{brassard:2005,
  Mermin1990}; the Deutsch-Jozsa games, where Alice and Bob
are given bit strings \(x\) and \(y\), and must output bit
strings \(a\) and \(b\) such that \(a = b\) if and only if
\(x = y\)~\cite{brassard:2005, Brassard_1999}; and, the
Magic Square game~\cite{brassard:2005, Mermin1990}. None of
these games admit a classical winning strategy (i.e.\ is not
possible to always win), yet they can be won systematically,
without any communication, provided that the players share
prior entanglement~\cite{brassard:2005}.

In this project, we shall explore the Mermin-Peres Magic
Square game. The origins of this game date back to the
ninities. It was first described ---~albeit not in the format of a
game~--- in the works of Mermin~\cite{mermin:1990} and of
Peres~\cite{peres:1990}. Their results provided (as per the
title of Mermin's review letter) a simplied proof for the
Kochen-Specker theorem~\cite{Kocher1975}. Later, in 2002,
Aravind demonstrated how to transform Mermin-Peres's proof
of then Kochen-Specker theorem into a proof of Bell's
theorem~\cite{aravind:xxx}. Bell's theorem shows the
incompatibility of hidden variables (i.e. determinism) and
locality, whereas Kochen-Specker's theorem demonstrates the
conflicting nature of determinism and non-contextuality.
These sort of results may seem challenging, but their
descriptions can be greatly simplified by modeling them as
quantum games.  In particular, it is extremely easy to show
that there cannot be a classical solution to the the Magic
Square game, and to convince an observer that something
``magical'' (classically impossible) is happening in a
successful implementation of a quantum winnin
strategy~\cite{brassard:2005}.

\paragraph{Outline of this paper.}
\section{Non-Locality and Contextuality}
\label{sec:local-context}


\section{Quantum Pseudo-Telepathy}
\label{sec:telepathy}

\section{The Magic Square Game}
\label{sec:magic-square}

\subsection{Classical Solution}
\label{sec:classic-sol}

\subsection{Quantum Solution}
\label{sec:quantum-sol}

\section{Quantum Unitaries (? research)}
\label{sec:quantum-unitaries}

\printbibliography{}
\end{document}
