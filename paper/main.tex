\documentclass{llncs}

%%%%%%%%%%%%%%%%%%%
%%% Packages
%%%%%%%%%%%%%%%%%%%
\usepackage{xcolor}
\usepackage[backend=biber, style=lncs]{biblatex}
\addbibresource{ref.bib}
% !TEX TS-program = pdflatexmk

%%%%%%%%%%%%%%%%%%%
%%% General Cmds
%%%%%%%%%%%%%%%%%%%
\newcommand{\todo}[1]{{\color{red}\bfseries{}TODO: #1}}
  
\begin{document}
\title{Quantum Pseudo-Telepathy}
\subtitle{The Magic Square Game}

\author{Luigi Soares \and Roberto Rosmaninho}
\institute{%
  Department of Computer Science, UFMG - Brazil \\
  \email{\{luigi.domenico, robertogomes\}@dcc.ufmg.br}}

\maketitle
% \begin{abstract}
% \end{abstract}

\section*{Proposal}
\label{sec:prop}

Telepathy, the ability of transmitting information from one person's
mind to another's, would certainly come in handy in many situations,
right? Unfortunately (or not), (to the best of our knowledge)
telepathy is not a thing. At least, not according to classical
physics. Certain aspects of the quantum realm, however, provide a way
of communication that for a layman looks as magical as ``true''
telepathy. This phenomenon is called quantum
\emph{pseudo-telepathy}~\cite{brassard:2005}.

Quantum pseudo-telepathy is observed in many contexts, usually
described in the format of a game: the ``impossible colouring games'',
based on the Kochen-Specker theorem~\cite{brassard:2005, Kochen1975}; the
parity games, in which \(n \geq 3\) players are given bit-strings and,
without communicating to each other, they output one of their bits,
winning if their outputs combined obey certain parity
conditions~\cite{brassard:2005, Mermin1990}; the Deutsch-Jozsa games, where
Alice and Bob are given bit strings \(x\) and \(y\), and must output
bit strings \(a\) and \(b\) such that \(a = b\) if and only if
\(x = y\)~\cite{brassard:2005, Brassard_1999}; and, the Magic Square game, the
topic of this research\cite{brassard:2005, Mermin1990}. None of these games
admit a classical winning strategy (i.e.\ is not possible to always
win), yet they can be won systematically, without any communication,
provided that the players share prior
entanglement\cite{brassard:2005}.

A \emph{Magic Square} is a \(3 \times 3\) matrix whose entries are
\{-1, 1\} (or sometimes \{0,~1\}), with the property that the sum of
each row is \(1\) and the sum of each column is \(-1\). The Magic
Square game is defined as follows: a referee Charlie asks Alice to
fill the entries of a row \(x \in \{0, 1, 2\}\), which he chooses
uniformly at random. Charlie, then, asks Bob to fill the entries of a
column \(y \in \{0, 1, 2\}\), also chosen at random. Alice and Bob win
if the product of the row given by Alice is \(1\), the product of the
column given by Bob is \(-1\) and the intersection of the given row
and column agrees.

A successful implementation of a quantum winning strategy for any
pseudo-telepathy game must convince an observer that something
classicaly impossible is happening\cite{brassard:2005}. Fortunately,
proving the classical impossibility of the Magic Square game is
extremely easy. Assuming a deterministic strategy, even if Alice and
Bob are allowed to talk \emph{before} the game starts, the best that
they can do is to construct two grids in advance, one for Alice and
another for Bob, and give the entries according to the matrices
that they prepared. However, the product of Alice's matrix will always
be \(1\) and the product of Bob's matrix will always be \(-1\), so
there must be one cell that does not agree. Probabilistic strategies
cannot do better than deterministic~\cite[Theorem 2]{brassard:2005}, so
the Magic Square game cannot be won under classical assumptions.

We will approach this project as follows:

\begin{enumerate}
\item Describe the quantum Magic Square game in details;
\item Discuss classical solutions: illustrate with examples and
  implement them (and make the code available through a jupyter
  notebook).
\item Demonstrate that no classical strategy can win with probability
  1. Show that the best that a classical strategy can achieve is
  is win 8 out of 9 rounds;
\item Discuss the quantum solution, using entanglement: illustrate
  with examples;
\item Demonstrate that the quantum strategy can win with probability
  1; and
\item Implement the quantum solution in Qiskit
  (and make the code available through a jupyter notebook).
\end{enumerate}

\printbibliography{}
\end{document}
