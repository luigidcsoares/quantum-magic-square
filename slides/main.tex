\documentclass[t, aspectratio=169, table]{beamer}

%%%%%%%%%%
%%% Packages
%%%%%%%%%%

\usepackage{xcolor}
\usepackage{hhline}
\usepackage{booktabs}
\usepackage{nicefrac}
\usepackage{tikz}
\usetikzlibrary{quantikz2}
\usepackage{tcolorbox}
\tcbuselibrary{theorems}

\newtcbtheorem{tcblemma}{Lema}{
  boxrule=0.15mm,
  colback=gray!15,
  % colbacktitle=gray!15,
  % coltitle=red!50!black,
  % fonttitle=\bfseries,
  coltext=red!40!black,
  detach title,
  % before upper={\tcbtitle},
  % separator sign none,
  % description delimiters={(}{). }
}{lem}

\newcommand{\ketbra}[2]{\ensuremath{| #1 \rangle \langle #2 |}}
\newcommand{\ci}{\mathrm{i}}
\DeclarePairedDelimiter{\card}{\lvert}{\rvert}
\DeclarePairedDelimiter{\norm}{\lVert}{\rVert}
\DeclareMathOperator{\tr}{tr}
\newcommand{\cnot}{\mathit{CNOT}}

%%%%%%%%%%
%%% Sections
%%%%%%%%%%

\newif{\ifsectionPage}
\newcommand{\sectionPageDefault}{\global\sectionPagetrue}
\newcommand{\disableNextSectionPage}{\global\sectionPagefalse}
\sectionPageDefault{}
\AtBeginSection[]{%
  \ifsectionPage{}
    \frame{\sectionpage}
  \fi
  \sectionPageDefault{}}

%%%%%%%%%%
%%% Doc Info
%%%%%%%%%%

\newcommand{\email}[1]{\href{mailto:#1}{#1}}
\author[Luigi Soares \& Roberto Rosmaninho]{%
  Luigi Soares (\email{luigi.domenico@dcc.ufmg.br}) \\
  Roberto Rosmaninho (\email{TODO})}
\institute[DCC/UFMG]{}
\date{03/07/2023}
\title[Pseudotelepatia Quântica]{Pseudotelepatia Quântica (TODO)}

\usetheme{ufmg}

\begin{document}
\maketitle

\section{Quadrado Mágico}

\begin{frame}{Introdução ao Jogo}
  \begin{itemize}[<+->]
  \item Jogo cooperativo não-local
  \item Dois jogadores: Alice e Bob
  \item Um árbitro: Charlie
  \item Alice e Bob não podem se comunicar após o início
  \item A cada rodada:
    \begin{enumerate}[<+->]
    \item Charlie sorteia uma linha 0, 1 ou 2 de uma matriz \(3 \times 3\) e atribui à Alice
    \item Alice preenche as três células com \(+1\) ou \(-1\), de forma que o produto seja \(+1\)
    \item Charlie sorteia uma coluna 0, 1 ou 2 da matriz e atribui ao Bob
    \item Bob preenche as três células com \(+1\) ou \(-1\), de forma que o produto seja \(-1\)
    \item Alice e Bob vencem se respeitaram  e concordaram
      no valor da interseção
    \end{enumerate}
  \end{itemize}
\end{frame}

\begin{frame}{Exemplo 1 (Vitória)}
  \renewcommand{\arraystretch}{2}
  \vfill{}
  \begin{minipage}{0.45\textwidth}
    \begin{center}
      Alice \(\leftarrow 0\)
    \end{center}
    \[\begin{array}{|>{\quad}c<{\quad}|*{2}{>{\quad}c<{\quad}|}c}
        \hhline{---}
        +1 & \cellcolor{gray!50} -1 & -1 & \prod = +1
        \\ \hhline{---}
           & & &
        \\ \hhline{---}
           & & &
        \\ \hhline{---}
        \multicolumn{1}{c}{}
        & \multicolumn{1}{c}{}
        & \multicolumn{1}{c}{}
      \end{array}\]
  \end{minipage}
  \hfill{}
  \begin{minipage}{0.45\textwidth}
    \begin{center}
      Bob \(\leftarrow 1\)
    \end{center}
    \[\begin{array}{|>{\quad}c<{\quad}|*{2}{>{\quad}c<{\quad}|}}
        \hhline{---}
        \phantom{+1} & \cellcolor{gray!50}-1 & \phantom{-1}
        \\ \hhline{---}
           & +1 &
        \\ \hhline{---}
           & +1 &
        \\ \hhline{---}
        \multicolumn{1}{c}{}
        & \multicolumn{1}{c}{\prod = -1}
        & \multicolumn{1}{c}{}
      \end{array}\]
  \end{minipage}  
  \vfill{}
\end{frame}

\begin{frame}{Exemplo 2 (Derrota, Interseção)}
  \renewcommand{\arraystretch}{2}
  \vfill{}
  \begin{minipage}{0.45\textwidth}
    \begin{center}
      Alice \(\leftarrow 1\)
    \end{center}
    \[\begin{array}{|>{\quad}c<{\quad}|*{2}{>{\quad}c<{\quad}|}c}
        \hhline{---}
           & & &
        \\ \hhline{---}
        +1 & \cellcolor{red!80!black!55} -1 & -1 & \prod = +1
        \\ \hhline{---}
           & & &
        \\ \hhline{---}
        \multicolumn{1}{c}{}
        & \multicolumn{1}{c}{}
        & \multicolumn{1}{c}{}
      \end{array}\]
  \end{minipage}
  \hfill{}
  \begin{minipage}{0.45\textwidth}
    \begin{center}
      Bob \(\leftarrow 1\)
    \end{center}
    \[\begin{array}{|>{\quad}c<{\quad}|*{2}{>{\quad}c<{\quad}|}}
        \hhline{---}
        \phantom{+1} & -1 & \phantom{-1}
        \\ \hhline{---}
           & \cellcolor{red!80!black!55} +1 &
        \\ \hhline{---}
           & +1 &
        \\ \hhline{---}
        \multicolumn{1}{c}{}
        & \multicolumn{1}{c}{\prod = -1}
        & \multicolumn{1}{c}{}
      \end{array}\]
  \end{minipage}  
  \vfill{}
\end{frame}

\begin{frame}{Exemplo 3 (Derrota, Produto)}
  \renewcommand{\arraystretch}{2}
  \vfill{}
  \begin{minipage}{0.45\textwidth}
    \begin{center}
      Alice \(\leftarrow 1\)
    \end{center}
    \[\begin{array}{|>{\quad}c<{\quad}|*{2}{>{\quad}c<{\quad}|}c}
        \hhline{---}
           & & &
        \\ \hhline{---}
        +1 & \cellcolor{gray!50} +1 & -1 & \cellcolor{red!80!black!55}\prod = -1
        \\ \hhline{---}
           & & &
        \\ \hhline{---}
        \multicolumn{1}{c}{}
        & \multicolumn{1}{c}{}
        & \multicolumn{1}{c}{}
      \end{array}\]
  \end{minipage}
  \hfill{}
  \begin{minipage}{0.45\textwidth}
    \begin{center}
      Bob \(\leftarrow 1\)
    \end{center}
    \[\begin{array}{|>{\quad}c<{\quad}|*{2}{>{\quad}c<{\quad}|}}
        \hhline{---}
        \phantom{+1} & -1 & \phantom{-1}
        \\ \hhline{---}
           & \cellcolor{gray!50} +1 &
        \\ \hhline{---}
           & +1 &
        \\ \hhline{---}
        \multicolumn{1}{c}{}
        & \multicolumn{1}{c}{\prod = -1}
        & \multicolumn{1}{c}{}
      \end{array}\]
  \end{minipage}  
  \vfill{}
\end{frame}

\begin{frame}{Estratégia Clássica (Determinística)}
  \begin{itemize}[<+->]
  \item Alice e Bob não podem se comunicar \emph{durante} o jogo, mas
    podem \emph{antes}
    
  \item Uma estratégia determinística consiste em 
    prepar matrizes pré-definidas
    
  \item<only@7-> É impossível vencer com 100\% de chance toda rodada
    
  \item<only@9-> A melhor estratégia determinística vence com probabilidade \(\nicefrac{8}{9}\)
  \end{itemize}

  \only<3-6>{
    \renewcommand{\arraystretch}{2}
    \vfill{}
    \begin{minipage}{0.45\textwidth}
      \begin{center}
        Alice
        \[\begin{array}{|>{\quad\columncolor{gray!50}}c<{\quad}|*{2}{>{\quad}c<{\quad}|}}
            \hhline{---}
            -1 & \onslide<4->{\only<5->{\cellcolor{gray!50}}-1} & \onslide<4->{\only<6->{\cellcolor{red!80!black!55}}+1}
            \\ \hhline{---}
            -1 & \onslide<4->{\only<5->{\cellcolor{gray!50}}-1} & \onslide<4->{\only<6->{\cellcolor{gray!50}}+1}
            \\ \hhline{---}
            -1 & \onslide<4->{\only<5->{\cellcolor{gray!50}}-1} & \onslide<4->{\only<6->{\cellcolor{gray!50}}+1}
            \\ \hhline{---}
          \end{array}\]
      \end{center}
    \end{minipage}
    \hfill{}
    \begin{minipage}{0.45\textwidth}
      \begin{center}
        Bob
        \[\begin{array}{|>{\quad\columncolor{gray!50}}c<{\quad}|*{2}{>{\quad}c<{\quad}|}}
            \hhline{---}
            -1 & \onslide<5->{\only<5->{\cellcolor{gray!50}}-1} & \onslide<6->{\only<6->{\cellcolor{red!80!black!55}}-1}
            \\ \hhline{---}
            -1 & \only<5->{\cellcolor{gray!50}-1} & \only<6->{\cellcolor{gray!50}+1}
            \\ \hhline{---}
            -1 & \only<5->{\cellcolor{gray!50}-1} & \only<6->{\cellcolor{gray!50}+1}
            \\ \hhline{---}
          \end{array}\]
      \end{center}
    \end{minipage}  
    \vfill{}
  }

  \only<8>{
    \[\begin{aligned}
        m_{0,0}\, m_{0,1}\, m_{0,2} &= +1 \\
        m_{1,0}\, m_{1,1}\, m_{1,2} &= +1 \\
        m_{2,0}\, m_{2,1}\, m_{2,2} &= +1 \\
        m_{0,0}\, m_{1,0}\, m_{2,0} &= -1 \\
        m_{0,1}\, m_{1,1}\, m_{2,1} &= -1 \\
        m_{0,2}\, m_{1,2}\, m_{2,2} &= -1 \\
        \midrule
        +1 &= -1
      \end{aligned}\]
  }
\end{frame}

\begin{frame}{Estratégia Clássica (Probabilística)}
  \begin{itemize}[<+->]
  \item Alice e Bob jogam uma moeda para decidir o valor de cada célula
    
  \item Isto é equivalente a cada um deles sortear uma
    dentre todas as possíveis matrizes
    
  \item A combinação das duas matrizes sorteadas é uma estratégia determinística
    
  \item Ou seja, qualquer estratégia probabilística é limitada pela
    melhor estratégia determinística. Logo, a chance de sucesso
    clássico é no máximo \(\nicefrac{8}{9}\)
  \end{itemize}
\end{frame}

\begin{frame}{Estratégia Quântica}
  \begin{itemize}[<only@+-5>]
  \item E se Alice e Bob puderem carregar qubits?
    
  \item Existe uma estratégia quântica que os permite vencer qualquer
    rodada

  \item Alice e Bob compartilham qubits emaranhados (um par para cada rodada):
    \[
      \ket{\psi} = \frac{1}{2}\left(
        \ket{00}_{A} \otimes \ket{00}_{B} + \ket{01}_{A} \otimes \ket{01}_{B} +
        \ket{10}_{A} \otimes \ket{10}_{B} + \ket{11}_{A} \otimes \ket{11}_{B}
      \right)
    \]
    
  \item Ao receber uma linha/coluna, eles medem seus qubits
    
  \item  O resultado de cada medição é o valor de cada célula
  \end{itemize}

  \only<+->{
    \renewcommand{\arraystretch}{2}
    \centering
    \[\begin{array}{|*{3}{>{\quad}c<{\quad}|}c}
        \hhline{---~}
        +I \otimes Z & +Z \otimes I & +Z \otimes Z & \onslide<.(1)->{\color{green!50!black}+I}
        \\ \hhline{---~}
        +X \otimes I & +I \otimes X & +X \otimes X & \onslide<.(1)->{\color{green!50!black}+I}
        \\ \hhline{---~}
        -X \otimes Z & -Z \otimes X & +Y \otimes Y & \onslide<.(1)->{\color{green!50!black}+I}
        \\ \hhline{---~}
        \multicolumn{1}{c}{\onslide<.(1)->{\color{red!50!black}-I}} &
        \multicolumn{1}{c}{\onslide<.(1)->{\color{red!50!black}-I}} &
        \multicolumn{1}{c}{\onslide<.(1)->{\color{red!50!black}-I}} &
      \end{array}\]
  }
\end{frame}

\begin{frame}{Exemplo 4}
  \begin{enumerate}[<only@+->]
  \item Alice recebe a linha 0 e mede seu segundo qubit com
    \(Z\). Suponha que ela tenha observado \(+1\). Ela atribui \(+1\)
    à primeira celula e o estado inicial \(\ket{\psi}\) colapsa para
    \[
      \ket{\psi_{A1}} = \frac{1}{\sqrt{2}} (\ket{0000} + \ket{1010})
    \]
  \item Em seguida, Alice mede seu primeiro qubit com \(Z\). Suponha que
    ela tenha observado \(-1\). Ela atribui \(-1\) à segunda célula e o estado
    colapsa para
    \[
      \ket{\psi_{A2}} = \ket{1010}
    \]
  \item Para a terceira célula, ela mede seus dois qubits com \(Z \otimes Z\).
    O único resultado possível é \(-1\) e o estado não se altera:
    \(
      \ket{\psi_{A3}} = \ket{\psi_{A2}}.
    \)
  \end{enumerate}
\end{frame}

\begin{frame}{Exemplo 4}
  \begin{enumerate}[<only@+->]
  \item Suponha que Bob tenha recebido a coluna 1. Para a primeira
    célula, ele mede seu primeiro qubit com \(Z\). Ele observa \(-1\) com
    (igual Alice) com probabilidade 1, e
    \[
      \ket{\psi_{B1}} = \ket{\psi_{A3}} = \ket{1010}
    \]
    
  \item Em seguida, Bob mede seu segundo qubit com \(X\).
    Suponha que ele observe \(+1\). Então, o estado colapsa para
    \[
      \ket{\psi_{B2}} = \frac{1}{\sqrt{2}} (\ket{1010} + \ket{1011})
    \]

  \item Para a terceira célula, ele mede ambos os qubits com
    \(-Z \otimes X\).  O único resultado possível é \(+1\), o que
    satisfaz a restrição sobre o produto da coluna.
  \end{enumerate}
\end{frame}

\begin{frame}{Lema 1: Produto das Linhas e Colunas}
  \begin{tcblemma}{}{msquare-quantum-prod}
    Seja \(M_{i, j}\) o observável na célula \((i, j)\) da estratégia
    quântica, segue que, para toda linha \(i\),
    \(\prod_{j} \operatorname{Out}(M_{i, j}) = +1\), e, para toda
    coluna \(j\), \(\prod_{i} \operatorname{Out}(M_{i, j}) = -1\).
  \end{tcblemma}

  \begin{itemize}[<+(1)->]
  \item Note que, em qualquer linha e coluna, os três
    observáveis comutam entre si.
    
  \item Ou seja, é possível encontrar uma base de autovetores em comum, que
    diagonaliza os três. Para toda linha \(i\) (equiv. coluna \(j\)),
    \begin{align*}
      M_{i,0} M_{i,1} M_{i,2} &= (P_{i} \Lambda_{i, 0}P^{-1}) (P_{i} \Lambda_{i, 1}P^{-1}) (P_{i} \Lambda_{i, 2}P_{i}^{-1}) \\
      &= P_{i} (\Lambda_{i, 0} \Lambda_{i, 1} \Lambda_{i, 2})P_{i}^{-1} \\
      &= P_{i} \Lambda_{i} P_{i}^{-1}
    \end{align*}
    
  \item Cada entrada de \(\Lambda_{i}\) é o produto dos autovalores p/ mesmo autovetor
  \end{itemize}
\end{frame}

\begin{frame}{Lema 1: Produto das Linhas e Colunas}
  \begin{itemize}[<+->]
  \item Os autovalores são sempre \(+1\) e \(-1\)
    
  \item Tem sempre dois autovetores em comum para cada autoespaço
    
  \item O que isso significa é que a primeira medição de qualquer
    linha/coluna resulta em um autovalor \(\lambda\) colapsa para uma
    combinação de dois autovetores \(\ket{\lambda}_{0}\) e
    \(\ket{\lambda}_{1}\)
    
  \item A segunda medição, então, colapsa para \(\ket{\lambda}_{0}\)
    ou \(\ket{\lambda_{1}}\), e fixa o estado.

  \item Logo, as medições irão resultar nos autovalores correspondentes
    a \(\ket{\lambda}_{0}\)  ou \(\ket{\lambda}_{1}\)

  \item O produto das três medições é uma das entradas de \(\Lambda_{i}\)
    
  \item Mas, para toda linha \(i\), temos
    \(P_{i} \Lambda_{i} P_{i}^{-1} = I\). Logo, \(\Lambda_{i} = I\) e o
    produto das três medições é \(+1\) sempre. Para toda coluna \(j\),
    \(P_{j} \Lambda_{j} P_{j}^{-1} = -I\) e \(\Lambda_{j} = -I\)
  \end{itemize}
\end{frame}
% \begin{center}
    %   \begin{tikzpicture}[vertex/.style={circle, thin, fill=blue!30, draw, inner sep=0, minimum size=5pt}]
    %     \node (phantom) [
    %     vertex, label=above right:{\phantom{\(-X \otimes Z\)}},
    %     label=below right:{\phantom{\(-X \otimes Z\)}}
    %     ] at (2, -4) {};
        
    %     \node (00) [vertex, label=above left:{\(I \otimes Z\)},] at (-2, 0) {};
    %     \node (01) [vertex, label=above left:{\(Z \otimes I\)}] at (0,  0) {};
    %     \node (02) [vertex, label=above left:{\(Z \otimes X\)}] at (2,  0) {};

    %     \node (10) [vertex, label=above left:{\(X \otimes I\)}] at (-2, -2) {};
    %     \node (11) [vertex, label=above left:{\(I \otimes X\)}] at (0,  -2) {};
    %     \node (12) [vertex, label=above left:{\(X \otimes X\)}] at (2,  -2) {};

    %     \node (20) [vertex, label=above left:{\(-X \otimes Z\)}] at (-2, -4) {};
    %     \node (21) [vertex, label=above left:{\(-Z \otimes X\)}] at (0,  -4) {};
    %     \node (22) [vertex, label=above left:{\(Y \otimes Y\)}] at (2,  -4) {};
        
    %     \only<.(1)->{\node (r0) [text=green!50!black!80] at (2.5, 0) {\(+I\)};}
    %     \only<.(1)->{\node (r1) [text=green!50!black!80] at (2.5, -2) {\(+I\)};}
    %     \only<.(1)->{\node (r2) [text=green!50!black!80] at (2.5, -4) {\(+I\)};}
        
    %     \only<.(1)->{\node (c0) [text=red!50!black!80] at (-2, -4.5) {\(-I\)};}
    %     \only<.(1)->{\node (c1) [text=red!50!black!80] at (0, -4.5) {\(-I\)};}
    %     \only<.(1)->{\node (c2) [text=red!50!black!80] at (2, -4.5) {\(-I\)};}

    %     \path[]
    %     (00) edge (01) edge (10)
    %     (01) edge (02) edge (11)
    %     (02) edge (12)
        
    %     (10) edge (11) edge (20)
    %     (11) edge (12) edge (21)
    %     (12) edge (22)

    %     (20) edge (21)
    %     (21) edge (22)
    %     ;
    %   \end{tikzpicture}
    % \end{center}
\end{document}
