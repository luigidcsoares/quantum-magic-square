\documentclass[t, aspectratio=169, table]{beamer}

%%%%%%%%%%
%%% Packages
%%%%%%%%%%

\usepackage{xcolor}
\usepackage{hhline}

%%%%%%%%%%
%%% Sections
%%%%%%%%%%

\newif{\ifsectionPage}
\newcommand{\sectionPageDefault}{\global\sectionPagetrue}
\newcommand{\disableNextSectionPage}{\global\sectionPagefalse}
\sectionPageDefault{}
\AtBeginSection[]{%
  \ifsectionPage{}
    \frame{\sectionpage}
  \fi
  \sectionPageDefault{}}

%%%%%%%%%%
%%% Doc Info
%%%%%%%%%%

\newcommand{\email}[1]{\href{mailto:#1}{#1}}
\author[Luigi Soares \& Roberto Rosmaninho]{%
  Luigi Soares (\email{luigi.domenico@dcc.ufmg.br}) \\
  Roberto Rosmaninho (\email{TODO})}
\institute[DCC/UFMG]{}
\date{03/07/2023}
\title[Pseudotelepatia Quântica]{Pseudotelepatia Quântica (TODO)}

\usetheme{ufmg}

\begin{document}
\maketitle

\section{Quadrado Mágico}

\begin{frame}{Introdução ao Jogo}
  \begin{itemize}[<+->]
  \item Jogo cooperativo não-local
  \item Dois jogadores: Alice e Bob
  \item Um árbitro: Charlie
  \item Alice e Bob não podem se comunicar após o início
  \item A cada rodada:
    \begin{enumerate}[<+->]
    \item Charlie sorteia uma linha 0, 1 ou 2 de uma matriz \(3 \times 3\) e atribui à Alice
    \item Alice preenche as três células com \(+1\) ou \(-1\), de forma que o produto seja \(+1\)
    \item Charlie sorteia uma coluna 0, 1 ou 2 da matriz e atribui ao Bob
    \item Bob preenche as três células com \(+1\) ou \(-1\), de forma que o produto seja \(-1\)
    \item Alice e Bob vencem se respeitaram  e concordaram
      no valor da interseção
    \end{enumerate}
  \end{itemize}
\end{frame}

\begin{frame}{Exemplo 1 (Vitória)}
  \renewcommand{\arraystretch}{2}
  \vfill{}
  \begin{minipage}{0.45\textwidth}
    \begin{center}
      Alice \(\leftarrow 0\)
    \end{center}
    \[\begin{array}{|>{\quad}c<{\quad}|*{2}{>{\quad}c<{\quad}|}c}
        \hhline{---}
        +1 & \cellcolor{gray!50} -1 & -1 & \prod = +1
        \\ \hhline{---}
           & & &
        \\ \hhline{---}
           & & &
        \\ \hhline{---}
        \multicolumn{1}{c}{}
        & \multicolumn{1}{c}{}
        & \multicolumn{1}{c}{}
      \end{array}\]
  \end{minipage}
  \hfill{}
  \begin{minipage}{0.45\textwidth}
    \begin{center}
      Bob \(\leftarrow 1\)
    \end{center}
    \[\begin{array}{|>{\quad}c<{\quad}|*{2}{>{\quad}c<{\quad}|}}
        \hhline{---}
        \phantom{+1} & \cellcolor{gray!50}-1 & \phantom{-1}
        \\ \hhline{---}
           & +1 &
        \\ \hhline{---}
           & +1 &
        \\ \hhline{---}
        \multicolumn{1}{c}{}
        & \multicolumn{1}{c}{\prod = -1}
        & \multicolumn{1}{c}{}
      \end{array}\]
  \end{minipage}  
  \vfill{}
\end{frame}

\begin{frame}{Exemplo 2 (Derrota, Interseção)}
  \renewcommand{\arraystretch}{2}
  \vfill{}
  \begin{minipage}{0.45\textwidth}
    \begin{center}
      Alice \(\leftarrow 1\)
    \end{center}
    \[\begin{array}{|>{\quad}c<{\quad}|*{2}{>{\quad}c<{\quad}|}c}
        \hhline{---}
           & & &
        \\ \hhline{---}
        +1 & \cellcolor{red!80!black!55} -1 & -1 & \prod = +1
        \\ \hhline{---}
           & & &
        \\ \hhline{---}
        \multicolumn{1}{c}{}
        & \multicolumn{1}{c}{}
        & \multicolumn{1}{c}{}
      \end{array}\]
  \end{minipage}
  \hfill{}
  \begin{minipage}{0.45\textwidth}
    \begin{center}
      Bob \(\leftarrow 1\)
    \end{center}
    \[\begin{array}{|>{\quad}c<{\quad}|*{2}{>{\quad}c<{\quad}|}}
        \hhline{---}
        \phantom{+1} & -1 & \phantom{-1}
        \\ \hhline{---}
           & \cellcolor{red!80!black!55} +1 &
        \\ \hhline{---}
           & +1 &
        \\ \hhline{---}
        \multicolumn{1}{c}{}
        & \multicolumn{1}{c}{\prod = -1}
        & \multicolumn{1}{c}{}
      \end{array}\]
  \end{minipage}  
  \vfill{}
\end{frame}

\begin{frame}{Exemplo 3 (Derrota, Produto)}
  \renewcommand{\arraystretch}{2}
  \vfill{}
  \begin{minipage}{0.45\textwidth}
    \begin{center}
      Alice \(\leftarrow 1\)
    \end{center}
    \[\begin{array}{|>{\quad}c<{\quad}|*{2}{>{\quad}c<{\quad}|}c}
        \hhline{---}
           & & &
        \\ \hhline{---}
        +1 & \cellcolor{gray!50} +1 & -1 & \cellcolor{red!80!black!55}\prod = -1
        \\ \hhline{---}
           & & &
        \\ \hhline{---}
        \multicolumn{1}{c}{}
        & \multicolumn{1}{c}{}
        & \multicolumn{1}{c}{}
      \end{array}\]
  \end{minipage}
  \hfill{}
  \begin{minipage}{0.45\textwidth}
    \begin{center}
      Bob \(\leftarrow 1\)
    \end{center}
    \[\begin{array}{|>{\quad}c<{\quad}|*{2}{>{\quad}c<{\quad}|}}
        \hhline{---}
        \phantom{+1} & -1 & \phantom{-1}
        \\ \hhline{---}
           & \cellcolor{gray!50} +1 &
        \\ \hhline{---}
           & +1 &
        \\ \hhline{---}
        \multicolumn{1}{c}{}
        & \multicolumn{1}{c}{\prod = -1}
        & \multicolumn{1}{c}{}
      \end{array}\]
  \end{minipage}  
  \vfill{}
\end{frame}

\begin{frame}{Estratégia Clássica (Determinística)}
  \begin{itemize}[<+->]
  \item Alice e Bob não podem se comunicar \emph{durante} o jogo, mas
    podem \emph{antes}
  \item Uma estratégia determinística consiste em 
    preparem matrizes pré-definidas
  \end{itemize}

  \only<+->{
    \renewcommand{\arraystretch}{2}
    \vfill{}
    \begin{minipage}{0.45\textwidth}
      \begin{center}
        Alice
        \[\begin{array}{|>{\quad\columncolor{gray!50}}c<{\quad}|*{2}{>{\quad}c<{\quad}|}}
            \hhline{---}
            -1 & \onslide<+->{\only<.(1)->{\cellcolor{gray!50}}-1} & \onslide<.->{\only<.(2)->{\cellcolor{red!80!black!55}}+1}
            \\ \hhline{---}
            -1 & \onslide<.->{\only<.(1)->{\cellcolor{gray!50}}-1} & \onslide<.->{\only<.(2)->{\cellcolor{gray!50}}+1}
            \\ \hhline{---}
            -1 & \onslide<.->{\only<.(1)->{\cellcolor{gray!50}}-1} & \onslide<.->{\only<.(2)->{\cellcolor{gray!50}}+1}
            \\ \hhline{---}
          \end{array}\]
      \end{center}
    \end{minipage}
    \hfill{}
    \begin{minipage}{0.45\textwidth}
      \begin{center}
        Bob
        \[\begin{array}{|>{\quad\columncolor{gray!50}}c<{\quad}|*{2}{>{\quad}c<{\quad}|}}
            \hhline{---}
            -1 & \onslide<+->{\only<.->{\cellcolor{gray!50}}-1} & \onslide<.(1)->{\only<.(1)->{\cellcolor{red!80!black!55}}-1}
            \\ \hhline{---}
            -1 & \only<.->{\cellcolor{gray!50}-1} & \only<.(1)->{\cellcolor{gray!50}+1}
            \\ \hhline{---}
            -1 & \only<.->{\cellcolor{gray!50}-1} & \only<.(1)->{\cellcolor{gray!50}+1}
            \\ \hhline{---}
          \end{array}\]
      \end{center}
    \end{minipage}  
    \vfill{}
  }
\end{frame}
\end{document}
